\chapter{Grundlagen}

\section{Netzwerk Schichtenmodell}
Netzwerk Kommunikation zwischen Anwendungen wird üblicherweise als Schichtenmodell beschrieben.  
Die unterste Schicht stellt dabei das physical layer. Das beschreibt die Physische Übertragung von Daten. 
Darüber sorgt die Sicherungsschicht für eine funktionierende Verbindung zwischen Endgeräten und den Übertragungsmedium. Auf dieser Schicht wird zum Beispiel das Ethernet Protokoll eingesetzt, das die übertragenen Daten auf Fehler überprüft und im Zweifel verwirft. 
Darunter kommt die Vermittlungsschicht, in der die Endgeräte Adressiert werden und Routing und Datenflusskontrolle gesteuert werden. Ein wichtiges Protokoll dieser Schicht ist das IP Protokoll.
5blabla
Bei einer Datenübertragung von einer Anwendung zu einer auf einen anderen Endgerät laufenden werden die Daten in durch die Schichten nach unten gereicht, wobei in jeder Schicht ein neuer Header erzeugt wird, der für die jeweilige Schicht wichtige Informationen enthält. Zum Beispiel werden die Daten von der Anwendung mit be 
triebsystemsabhängigen Systemaufrufen an den TCPStack übergeben. Dieser erzeugt ein TCP Paket das außer den Daten einen Header enthält, welcher Information bereitstellt, die unter anderen für das richtige Zusammensetzen der einzelnen Datenpakte beim Empfänger, als auch für die Zuordnung der übertragen Daten zu der jeweiligen Anwendung benötigt werden. Bei der anschließenden Erzeugung des IP Pakets bilden das TCP Paket bestehend aus Daten und TCP-Header die zu übertragenden Daten. Der IP Header enthält unter anderen die IP-Adressen des Ziel und Quell Geräts. Das IP Paket bleibt im Normalfall unverändert, bis das Zielgerät erreicht ist. 

\section{Ethernet}
\section{IP}
\section{TCP}
\section{DHCP}
\section{AMIDAR}