\chapter{Grundlagen}

\section{Netzwerk Schichtenmodell}
Netzwerk Kommunikation zwischen Anwendungen wird üblicherweise als Schichtenmodell beschrieben.  
Die unterste Schicht stellt dabei das physical layer. Das beschreibt die Physische Übertragung von Daten. 
Darüber sorgt die Sicherungsschicht für eine funktionierende Verbindung zwischen Endgeräten und den Übertragungsmedium. Auf dieser Schicht wird zum Beispiel das Ethernet Protokoll eingesetzt, das die übertragenen Daten auf Fehler überprüft und im Zweifel verwirft. 
Darunter kommt die Vermittlungsschicht, in der die Endgeräte Adressiert werden und Routing und Datenflusskontrolle gesteuert werden. Ein wichtiges Protokoll dieser Schicht ist das IP Protokoll.



\section{Ethernet}
\section{IP}
\section{TCP}
\section{DHCP}
\section{AMIDAR}